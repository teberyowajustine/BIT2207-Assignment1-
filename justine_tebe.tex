\documentclass{article}

\usepackage[margin=1in]{geometry}

\begin{document}

		\title{A REPORT ABOUT CAR PARKING SYSTEM}
		\author{Author :  TEBERYOWA  JUSTINE }
		\date{Student No: 215014710}
		\maketitle
	

	\tableofcontents

	\section{Abstruct}
The chapter is to explore the product of the research conducted on existing car parking systems. The scope of this chapter is basically to identify some car parking system and by assigning the cars nearby.
Car parkingsystem is being used in many congested area or location where there are many meeting point of people like where there are more than one shopping complex near to each other or where there is megamall or stadium

\section{Introduction}
      A car parking system is a mechanical device that multiplies parking capacity inside a parking lot. Parking systems are generally powered by electric motors or hydraulic pumpsthat move vehicles into a storage position.

\subsection{Purpose}
     They offer convenience for vehicle users and efficient usage of space for urban-based companies. car park systems save space and 

\subsection{Purpose}
     They offer convenience for vehicle users and efficient usage of space for urban-based companies. car park systems save space and 

\subsection{Limitations}
The  system do not assign car to a specific parking lot and this result in roaming of cars inside the area in searching of parking space.

\subsection{Background}
In the 1920s, forerunners of automated parking systems appeared in U.S. cities like Los Angeles, Chicago, New York and Cincinnati. Some of these multi-story structures are still standing, and have been adapted for new uses.

\section{Procedures and observation}
   Enable the driver to collect ticket upon entrance: car Parking system allow the driver to get his ticket after he press the button of the gate barrier.
The system records the entire cars that pass through the entrance.
The customer presses the button on the machine; ticket will come out from it and the customer take his ticket and then the gate will open.

\section{Summary of results}
     From the research and comparisons being conducted by the author on several types of the car parking systems, the author have come to understand clearer the criteria used to judge and rank parking systems online.

\section{Conclusion}
   The problems being identified by the author in the existing systems has made clear what mistakes should be avoided in this project to make it unique, acceptable, and user friendly.


\end{document}